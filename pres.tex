\documentclass[aspectratio=169]{beamer}
\usepackage[utf8]{inputenc}
\usepackage{hyperref}
\usetheme{Boadilla}

\title{MIS40530 Reading Assignment}
\subtitle{Ant Colony Optimization: A New Meta-Heuristic}
\author{Adedayo Adekowolan and Sean Tully}
\institute{UCD}
\date{6\textsuperscript{th} October 2016}

\begin{document}

    \begin{frame}
        \titlepage
    \end{frame}

    \begin{frame}{Introduction and Context}
        The paper has two main aims:
        \begin{enumerate}
            \item Provide an overview of ant colony optimisation as a field and detail current research trends
            \item Introduce the \emph{Ant Colony Optimization} (ACO) meta-heuristic
        \end{enumerate}
    \end{frame}

    \begin{frame}{Gap and Research Aim}
        \begin{itemize}
            \item No clear gap is identified.
            \item The authors hope that the ACO meta-heuristic can be used to characterise ant colony algorithms and thus aid future research.
        \end{itemize}
    \end{frame}

    \begin{frame}{Methodology}
        \begin{enumerate}
            \item The parameters governing a generalised problem and solution are defined
            \item A set of rules are defined, governing the behaviour of the ants
            \item An implementation is provided, in pseudocode, of the ACO meta-heuristic
        \end{enumerate}
    \end{frame}

    \begin{frame}{Results}
        As a proof-of-concept, details are given on the application of the ACO meta-heuristic to some discrete optimisation problems:
        \begin{itemize}
            \item The traveling salesman problem
            \item The problem of routing in communications networks
        \end{itemize}
        \vspace{0.125in}
        The success (of others) in using ant colony algorithms to solve discrete optimisation problems is presented:
        \begin{itemize}
            \item Fast (comparable in speed to other state-of-the-art methods)
            \begin{itemize}
                \item The traveling salesman problem
                \item The quadratic assignment problem
                \item The sequential ordering problem
            \end{itemize}
            \item Slower than other methods
            \begin{itemize}
                \item The job scheduling problem (slower than other methods)
            \end{itemize}
        \end{itemize}
    \end{frame}

    \begin{frame}{Analysis and Signifiance}
        \begin{itemize}
            \item No quantitative results are presented
            \item The application examples show qualitatively that the ACO meta-heuristic can be applied to various problems
        \end{itemize}        
    \end{frame}

    \begin{frame}{Significance}
        \begin{itemize}
            \item The paper was published in conference proceedings
            \item It has 348 citations to date (\href{http://ieeexplore.ieee.org/document/782657/   }{IEEE Xplore})
            \item A more through paper was published in the same year in the the journal ``Artificial Life'', which has 3135 citations to date on Google Scholar
        \end{itemize}
    \end{frame}
    
    \begin{frame}{Conclusion}
        \begin{itemize}
            \item The paper was concise and well written
            \item The examples highlight the application of the algorithm
            \item The effectiveness of ant optimisation methods by others cannot be seen as an endorsement of the ACO meta-heuristic
            \item The level of citations indicate that the heuristic did prove to be of use in categorising ant colony algorithms
        \end{itemize}
    \end{frame}
\end{document}


